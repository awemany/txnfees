\documentclass{article}
\usepackage{hyperref}
\usepackage{times}
\usepackage{graphicx}
\usepackage{microtype}
\usepackage{afterpage}
\title{Transaction fee extrapolation}
\author{awemany (reddit, bitco.in)}
\begin{document}
\maketitle
\begin{section}{Overview}
  Olivier Janssens put up a reward of up to 10
  BTC\footnote{\url{https://www.reddit.com/r/btc/comments/57anod/can_someone_create_a_transaction_fee_income_graph/}}
  for producing an extrapolation of transaction income for miners.  This is an
  entry to this contest.

  This context is held in the wider context of convincing Bitcoin miners to
  switch their mining hardware to alternative Bitcoin implementations that
  support on-chain scaling, such as Bitcoin Unlimited.

  On-chain is believed by its proponents (such as the author of this document)
  to be a key element for Bitcoin's continued success.

  \emph{Disclaimer:} It should be noted that -as all predictions are- they can
  be wildly inaccurate and the author takes no responsibility for using the
  data and predictions shown herein. Predictions including exponentials even
  more so!\footnote{The author does not question the fact that exponentials
    can \emph{not} go on forever for physical reasons; however the author also
    assumes that we still have lots of \emph{potential} exponential growth
    ahead!}  There are also no guarantees that the calculations and estimates
  done here are even correct. However, the code and data is all out in the
  open, and anyone is invited to check it all.
\end{section}
\begin{section}{Assumptions and method}
  All data that is used for creating this document has been taken from the
  charts page of \url{http://blockchain.info/en/charts}. The contained script
  \texttt{get.sh} shows all commands to fetch an up-to-date set of data, as
  comma-separated-value (CSV) files.

  For this analysis, data ranging from March, 1st 2009 to Oct, 15th 2016 is
  used. This corresponds to the \texttt{.csv} files included in the
  repository.

  Data analysis is done using \texttt{python} (version 2) and its most common
  data analysis packages: \texttt{numpy}, \texttt{scipy} and \texttt{pandas}.
  
  The overall method to reach a graph of transaction income first looks at the
  historically close relation between transaction rate and market
  cap(italization) (MCAP). From this, a fit is used to estimate market cap
  from transaction rate. The author claims in no way novel insights on this
  relation. It has been prominently been demonstrated by Peter Rizun before.
  
  It should be noted that, although, so far, the famous $x^2$ relation holds
  to a high degree, the correlation (that is presumed to be an indication of
  Metcalfe's law about the value of scaling networks) is expected to break
  down at some unknown point in the future.
  
  This analysis and extrapolation further assumes that miners will think in
  dollars instead of BTC for the foreseeable future, and that the prices of
  equipment etc. are stable, when denominated in \$ (no inflation).
  
  It is a common complaint of those intending to scale Bitcoin onchain that
  onchain transaction volume is being crippled by the 1MB block size limit in
  place. This crippling can also be clearly seen in the graphs below. For the
  sake of this analysis, Bitcoin's history so far is divided into three periods:

  \begin{itemize}
  \item[\emph{Initial stage}]The early stage of growth, until Jan 2013
  \item[\emph{Early stage}]The time range from Jan 2013 to Dec 2015
  \item[\emph{Saturation stage}]The time from Dec 2015 till now, where it is assumed
    that the 1MB has an effect upon maximum transaction rate, and more
    importantly, also the actions of market participants.
    \end{itemize}
  

  Blockchain.info supplies a data set called the \emph{number of transactions
    excluding popular addresses} (NTEP). The excellent correlation between
  market cap and transaction rate is especially visible when relating to NTEP.

  As NTEP and total number of transactions seem to converge over time, it is
  assumed that NTEP is as good as regular total transaction rate for all
  analyses below, while avoiding the singular cases of Satoshi Dice and
  similar to influence the MCAP/TXN relation. NTEP excludes the 100 most
  popular addresses (measured by their number of outputs) from the total
  transaction rate.
\clearpage
\end{section}
\begin{section}{Transaction rate and market cap}
  \begin{figure}[h]
    \includegraphics[width=12cm]{../mcap-vs-txn-fullrange.pdf}
    \caption{MCAP, NTEP and total transaction rate.\label{fig:mcap-vs-txn-fullrange}}
    \end{figure}
  It has prominently been demonstrated by Peter Rizun before that MCAP and NTEP
  align if $\mathrm{NTEP} \times \mathrm{NTEP}$ is plotted on top of MCAP (in
  dollars). In Fig.~\ref{fig:mcap-vs-txn-fullrange}, this figure is recreated
  once more, using recent data. It also includes the NTEP as well as the total
  number of transactions.

  It can be seen that, as stated above, NTEP and total transaction count
  converge.  Also visible is that in the \emph{Saturation stage}, the MCAP
  seems to be suppressed compared to transaction count, presumably due to
  market actors factoring in the apparent unwillingness of many Bitcoin miners
  to scale Bitcoin.

  \begin{figure}[h]
    \includegraphics[width=12cm]{../mcap-vs-txn-scatter-and-fit.pdf}
    \caption{MCAP vs. NTEP and fit.\label{fig:mcap-vs-txn-scatter-and-fit}}
  \end{figure}
  
  The relation of \begin{equation}
    \mathrm{MCAP / \$} = \mathrm{NTEP} ^ 2
    \end{equation}
  seems to be a close fit already. In
  Fig.~\ref{fig:mcap-vs-txn-scatter-and-fit}, this relation is displayed
  once more, but in a different way.  Here the data from the \emph{Initial}
  and \emph{Early} stage is plotted as green dots, and that of the
  \emph{Saturation} stage in red.

  The blue line is a linear fit of the green data, of the form
  \begin{equation}
  \log(\mathrm{MCAP / \$}) = x \log \mathrm{NTEP} 
  \end{equation}

  yielding $x=2.03$.

  Clearly visible is the saturation and divergence from the squared behavior
  for the latest \emph{Saturation} period, where the blocksize limit has a
  profound effect upon Bitcoin.

\clearpage
\end{section}
\begin{section}{Transaction rate models}
  \begin{figure}[h]
    \includegraphics[width=12cm]{../txngrowth.pdf}
    \caption{Transaction rate modeling. \label{fig:txngrowth}}
  \end{figure}
  
  The next step to get an estimate of future transaction rates is to look at
  past behavior and try to extrapolate from it. In Fig.~\ref{fig:txngrowth},
  the daily number of transactions (blue curve) is plotted, plus three
  scenarios for further growth of this rate.

  Green is a very optimistic scenario, and is a linear fit of Bitcoin's
  transaction rate over the \emph{Initial stage} as well as the
  \emph{Early Stage} (thus yielding an exponential curve for transaction rate
  growth). It is highly unlikely that Bitcoin's growth will ever exceed this
  growth again. This curve would those be the most optimistic miner's
  expectation, should an open-ended blocksize be implemented soon.
  
  Red is the saturation scenario, keeping transaction rate at 1MB, as
  the indicated direction by several members of the Bitcoin Core team. This red
  line assumes that the maximum number of transactions per day that have ever
  been seen is also going to be a good approximation of the future number of
  transactions for all times.
  
  Black is a (in the near term) optimistic scenario that assumes that
  Bitcoin's transaction rate with will continue to grow for the near future
  with the same rate as it did in the \emph{Early}, but not the faster
  \emph{Initial} stage.

  It should be noted that these exponentials can not go on forever, or even a
  long time - world GDP is currently (Oct
  2016) \footnote{\url{https://en.wikipedia.org/wiki/Gross_world_product}}
  approximately $\$10^{14}$. Bitcoin taking over the world would thus happen
  in the middle of next year, or 2021 in the slightly less optimstic
  scenario. Both cases deemed highly unlikely by the author.
  
\clearpage
\end{section}
\begin{section}{Fee models}
  \begin{figure}
    \includegraphics[width=12cm]{../fee-income-models.pdf}
    \caption{Fee modeling.\label{fig:fee-income-models}}
  \end{figure}
  
  Using another leap of faith, the above transaction growth model can be
  extended into a model for the miner's income. The result of this can be seen
  in Fig.~\ref{fig:fee-income-models}.

  Using the flat transaction rate of the 1MB case and the two exponential
  models from the preceding section, and further using the MCAP - NTEP
  relation from above, total miner income can be estimated.
  
  \paragraph{Transaction cost.}{To create this extrapolation,
    is assumed that, for high on-chain volume scaling, a transaction will cost
    $\$0.02$ for all times. For the saturated case, both an assumed $\$0.02$
    as well as a high price of $\$50$ per transaction is plotted.}
  
  The MCAP and thus coin price is assumed to follow
  $\mathrm{MCAP}=\mathrm{NTEP}^{2.03}$ with transaction rate of for the respective
  model.

  As the above Fig.~\ref{fig:mcap-vs-txn-fullrange} shows, the market likely
  factored the unwillingness of many Bitcoiners to scale Bitcoin into the
  price. This means that the transaction rate growth until saturation, as
  observed in the \emph{Saturation stage} would reflect in a higher expected
  Bitcoin price even when being stuck with an 1MB limit forever. This higher
  price amounts to $\approx 5200$ \$/BTC, and an $\mathrm{MCAP}\approx 110
  \cdot \$ 10^9 $. For both the red and light red curve, this higher, constant
  MCAP is assumed.

  A further, minor point is that it is further assumed that the coin schedule
  is aligned to the last halving with exactly 4 years of halving time, in both
  directions of time. As this will not fit the past issuance schedule (which
  was sped up due to strong hash rate growth), an initial number of coins is
  set so that the number of coins at the first halving matches what is
  expected according to schedule. This should not impact later estimates at
  all. It is visible as a slight bend of the extrapolated green and black lines
  on the left side in this last plot.

  That the lines appear jagged is due to the dropping block reward.
  
  In this plot, the dark red line is assuming 1MB saturation and corresponding
  market cap, with a cost of $\$0.02$ per transaction. The light red line
  assumes an 1MB cap and a much higher cost of $\$50$ per transaction, in case
  Bitcoin is used primarily as a settlement layer.
  
  The green and black line are the exponential models for the \emph{Initial}
  and \emph{Initial and Early} stage growth of transaction volume, with
  corresponding market cap growth. Both lines assume a low, constatnt cost of
  $\$0.02$ per transaction on chain.

  The more optimistic, green line exceeded both red cases a while
  ago. Approximately today, also the $\$50$ high fee, but satured Bitcoin case
  is exceeded by an assumed NTEP and MCAP growth.
\clearpage
\end{section}

\begin{section}{Conclusions}
  In Fig.~\ref{fig:mcap-vs-txn-fullrange}, the black curve exceeds the light
  red curve at about this time.
  
  This means that even with an assumption of very high transaction prices in the forced
  saturation scenario of $\$50/\mathrm{txn}$ (settlement layer) and even
  further assuming that the growth as indicated by the black line continues only until
  2018, it would, according to this model, still be much better for miners to
  allow on-chain growth beyond 1MB.

  The future is uncertain. The author makes no claim that exponential growth
  will continue forever, but is certain that intentional crippling of
  Bitcoin's scapability to scale onchain is \emph{not} going to help Bitcoin
  get any more widespread adoption.

  Furthermore, it should be noted that one of the main arguments of the
  \emph{small blockist} is that higher level scaling solutions on top of
  Bitcoin will increase Bitcoin's market cap as well, which might bend the red
  curve upwards by an unknown amount. The time is now, though, and reliable,
  usable and widely accepted solutions have not been implemented yet.

  There is also no reason that higher level solutions on top of Bitcoin can not
  work synergistically with a lifted maximum block size limit.
\clearpage
\end{section}
\end{document}
